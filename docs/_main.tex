%%%%%%%%%%%%%%%%%%%%%%%%%%%%%%%%%%%%%%%%%%%%%%%%%%%%%%%%%%%%%%%
%% OXFORD THESIS TEMPLATE

% Use this template to produce a standard thesis that meets the Oxford University requirements for DPhil submission
%
% Originally by Keith A. Gillow (gillow@maths.ox.ac.uk), 1997
% Modified by Sam Evans (sam@samuelevansresearch.org), 2007
% Modified by John McManigle (john@oxfordechoes.com), 2015
% Modified by Ulrik Lyngs (ulrik.lyngs@cs.ox.ac.uk), 2018, for use with R Markdown
%
% Ulrik Lyngs, 25 Nov 2018: Following John McManigle, broad permissions are granted to use, modify, and distribute this software
% as specified in the MIT License included in this distribution's LICENSE file.
%
% John tried to comment this file extensively, so read through it to see how to use the various options.  Remember
% that in LaTeX, any line starting with a % is NOT executed.  Several places below, you have a choice of which line to use
% out of multiple options (eg draft vs final, for PDF vs for binding, etc.)  When you pick one, add a % to the beginning of
% the lines you don't want.


%%%%% CHOOSE PAGE LAYOUT
% The most common choices should be below.  You can also do other things, like replacing "a4paper" with "letterpaper", etc.

% This one will format for two-sided binding (ie left and right pages have mirror margins; blank pages inserted where needed):
%\documentclass[a4paper,twoside]{templates/ociamthesis}
% This one will format for one-sided binding (ie left margin > right margin; no extra blank pages):
%\documentclass[a4paper]{ociamthesis}
% This one will format for PDF output (ie equal margins, no extra blank pages):
%\documentclass[a4paper,nobind]{templates/ociamthesis}
%UL 2 Dec 2018: pass this in from YAML
\documentclass[a4paper, nobind]{templates/ociamthesis}
\usepackage{mathptmx}

% UL 30 Nov 2018 pandoc puts lists in 'tightlist' command when no space between bullet points in Rmd file
\providecommand{\tightlist}{%
  \setlength{\itemsep}{0pt}\setlength{\parskip}{0pt}}
 
% UL 1 Dec 2018, fix to include code in shaded environments

%UL 2 Dec 2018 reduce whitespace around verbatim environments
\usepackage{etoolbox}
\makeatletter
\preto{\@verbatim}{\topsep=0pt \partopsep=0pt }
\makeatother

%UL 26 Mar 2019, enable strikethrough
\usepackage[normalem]{ulem}

%UL 15 Oct 2019, enable link highlighting to be turned off from YAML
\definecolor{darkblue}{rgb}{0, 0, 0.5}
\usepackage[pdfpagelabels,
    colorlinks=true,
    citecolor=darkblue,
    filecolor=darkblue,
    urlcolor=darkblue,
    hidelinks=]{hyperref}
\hypersetup{
    colorlinks=true,
    citecolor=darkblue,
    filecolor=darkblue,
    urlcolor=darkblue,
    linkcolor=black,
}

%%%%% SELECT YOUR DRAFT OPTIONS
% Three options going on here; use in any combination.  But remember to turn the first two off before
% generating a PDF to send to the printer!

% This adds a "DRAFT" footer to every normal page.  (The first page of each chapter is not a "normal" page.)

% This highlights (in blue) corrections marked with (for words) \mccorrect{blah} or (for whole
% paragraphs) \begin{mccorrection} . . . \end{mccorrection}.  This can be useful for sending a PDF of
% your corrected thesis to your examiners for review.  Turn it off, and the blue disappears.
\correctionstrue

%%%%% BIBLIOGRAPHY SETUP
% Note that your bibliography will require some tweaking depending on your department, preferred format, etc.
% The options included below are just very basic "sciencey" and "humanitiesey" options to get started.
% If you've not used LaTeX before, I recommend reading a little about biblatex/biber and getting started with it.
% If you're already a LaTeX pro and are used to natbib or something, modify as necessary.
% Either way, you'll have to choose and configure an appropriate bibliography format...

% The science-type option: numerical in-text citation with references in order of appearance.
% \usepackage[style=numeric-comp, sorting=none, backend=biber, doi=false, isbn=false]{biblatex}
% \newcommand*{\bibtitle}{References}

% The humanities-type option: author-year in-text citation with an alphabetical works cited.
% \usepackage[style=authoryear, sorting=nyt, backend=biber, maxcitenames=2, useprefix, doi=false, isbn=false]{biblatex}
% \newcommand*{\bibtitle}{Works Cited}

%UL 3 Dec 2018: set this from YAML in index.Rmd
\usepackage[style=authoryear, sorting=nyt, backend=biber, maxcitenames=1, mincitenames=1, maxbibnames=100, minbibnames=100, useprefix, doi=false, url=false, isbn=false, uniquename=false]{biblatex}
\newcommand*{\bibtitle}{\textbf{Bibliography}}

% Embed reference URL under the title inside bibliography
\newbibmacro{string+url}[1]{%
  \iffieldundef{url}{#1}{\href{\thefield{url}}{#1}}
}
\DeclareFieldFormat{title}{\usebibmacro{string+url}{\mkbibemph{#1}}}
\DeclareFieldFormat*{title}{\usebibmacro{string+url}{\mkbibquote{#1}}}

% Comma between author and year in citations
\renewcommand*{\nameyeardelim}{\addcomma\space}

% This makes the bibliography left-aligned (not 'justified') and slightly smaller font.
\renewcommand*{\bibfont}{\raggedright\small}

% Change this to the name of your .bib file (usually exported from a citation manager like Zotero or EndNote).
\addbibresource{../references.bib}


% Uncomment this if you want equation numbers per section (2.3.12), instead of per chapter (2.18):
%\numberwithin{equation}{subsection}


%%%%% THESIS / TITLE PAGE INFORMATION
% Everybody needs to complete the following:
\title{Application of GLM improvements\\
on Non Life Insurance Pricing}
\author{Leonardo Stincone}
\universityname{Università degli Studi di Trieste}
\departmentname{Dipartimento di Scienze Economiche, Aziendali,\\
Matematiche e Statistiche}
\degreedef{Tesi di Laurea Magistrale}
\degreename{Laurea Magistrale in Scienze Statistiche e Attuariali}
\degreeclass{LM-83}
\academicyear{Anno Accademico 2019 - 2020}
\degreedate{March 2021}
\advisorname{Prof.~Francesco Pauli}
%\coadvisorname{}
\degreedate{March 2021}

%%%%% YOUR OWN PERSONAL MACROS
% This is a good place to dump your own LaTeX macros as they come up.

% To make text superscripts shortcuts
	\renewcommand{\th}{\textsuperscript{th}} % ex: I won 4\th place
	\newcommand{\nd}{\textsuperscript{nd}}
	\renewcommand{\st}{\textsuperscript{st}}
	\newcommand{\rd}{\textsuperscript{rd}}

%%%%% THE ACTUAL DOCUMENT STARTS HERE
\begin{document}

%%%%% CHOOSE YOUR LINE SPACING HERE
% This is the official option.  Use it for your submission copy and library copy:
\setlength{\textbaselineskip}{18pt plus2pt}
% This is closer spacing (about 1.5-spaced) that you might prefer for your personal copies:
%\setlength{\textbaselineskip}{18pt plus2pt minus1pt}

% You can set the spacing here for the roman-numbered pages (acknowledgements, table of contents, etc.)
\setlength{\frontmatterbaselineskip}{16pt plus1pt minus1pt}

% UL: You can set the line and paragraph spacing here for the separate abstract page to be handed in to Examination schools
\setlength{\abstractseparatelineskip}{13pt plus1pt minus1pt}
\setlength{\abstractseparateparskip}{0pt plus 1pt}

% UL: You can set the general paragraph spacing here - I've set it to 2pt (was 0) so
% it's less claustrophobic
\setlength{\parskip}{2pt plus 1pt}


% Leave this line alone; it gets things started for the real document.
\setlength{\baselineskip}{\textbaselineskip}


%%%%% CHOOSE YOUR SECTION NUMBERING DEPTH HERE
% You have two choices.  First, how far down are sections numbered?  (Below that, they're named but
% don't get numbers.)  Second, what level of section appears in the table of contents?  These don't have
% to match: you can have numbered sections that don't show up in the ToC, or unnumbered sections that
% do.  Throughout, 0 = chapter; 1 = section; 2 = subsection; 3 = subsubsection, 4 = paragraph...

% The level that gets a number:
\setcounter{secnumdepth}{3}
% The level that shows up in the ToC:
\setcounter{tocdepth}{3}


%%%%% ABSTRACT SEPARATE
% This is used to create the separate, one-page abstract that you are required to hand into the Exam
% Schools.  You can comment it out to generate a PDF for printing or whatnot.
\begin{abstractseparate}
  This is my abstract \dots
\end{abstractseparate}

% JEM: Pages are roman numbered from here, though page numbers are invisible until ToC.  This is in
% keeping with most typesetting conventions.
\begin{romanpages}

% Title page is created here
\maketitle

% Comment if you don't need a white page between title page and next pages
\null\newpage

% Default spacing for non-quote chapters and sections:
% 0 space before, 40pt space after title
% set at beginning of individual chapters where needed
\titlespacing*{\chapter}{0pt}{0pt}{35pt}

%%%%% DEDICATION -- If you'd like one, un-comment the following.

%%%%% ACKNOWLEDGEMENTS -- Nothing to do here except comment out if you don't want it.
\begin{acknowledgements}
 	Qua potrei ringraziare:

  \begin{itemize}
  \item
    Genertel per avermi dato i dati e supportato in questa ricerca
  \item
    DEAMS per la formazione e l'apertura alla collaborazione col mondo aziendale
  \end{itemize}
\end{acknowledgements}

%%%%% ABSTRACT -- Nothing to do here except comment out if you don't want it.
\begin{abstract}
  This is my abstract \dots
\end{abstract}

%%%%% MINI TABLES
% This lays the groundwork for per-chapter, mini tables of contents.  Comment the following line
% (and remove \minitoc from the chapter files) if you don't want this.  Un-comment either of the
% next two lines if you want a per-chapter list of figures or tables.

% This aligns the bottom of the text of each page.  It generally makes things look better.
\flushbottom

% This is where the whole-document ToC appears:
\renewcommand{\contentsname}{\textbf{Table of Contents}}
\renewcommand{\listfigurename}{\textbf{List of Figures}}
\renewcommand{\listtablename}{\textbf{List of Tables}}
\tableofcontents

\listoffigures
	\mtcaddchapter
  	% \mtcaddchapter is needed when adding a non-chapter (but chapter-like) entity to avoid confusing minitoc

% Uncomment to generate a list of tables:
\listoftables
  \mtcaddchapter
%%%%% LIST OF ABBREVIATIONS
% This example includes a list of abbreviations.  Look at text/abbreviations.tex to see how that file is
% formatted.  The template can handle any kind of list though, so this might be a good place for a
% glossary, etc.
% Leonardo Stincone, 18/04/2021
% This code must be included in the preamble
% Package needed for abbreviations
\usepackage{acro}

% Defining the abbreviations
\DeclareAcronym{glm}{
  short = GLM,
  long = Generalized Linear Model,
}

\DeclareAcronym{aic}{
  short = AIC,
  long = Akaike Information Criterion,
}

\DeclareAcronym{bic}{
  short = BIC,
  long = Bayesian Information Criterion,
}

\DeclareAcronym{gam}{
  short = GAM,
  long = Generalized Additive Model,
}

\DeclareAcronym{map}{
  short = MAP,
  long = Maximum a Posteriori,
}

\DeclareAcronym{gbm}{
  short = GBM,
  long = Gradient Boosting Machine,
}

\DeclareAcronym{lasso}{
  short = LASSO,
  long = Least Absolute Shrinkage and Selection Operator,
}

\DeclareAcronym{rf}{
  short = RF,
  long = Random Forest,
}

\DeclareAcronym{nn}{
  short = NN,
  long = Neural Network,
}

\DeclareAcronym{mtpl}{
  short = MTPL,
  long = Motor Third Party Liability,
}

\DeclareAcronym{mod}{
  short = MOD,
  long = Motor Own Damage,
}

\DeclareAcronym{ibnyr}{
  short = IBNyR,
  long = Incurred But Not yet Reported claim,
}

\DeclareAcronym{ibner}{
  short = IBNeR,
  long = Incurred But Not enough Reported claim,
}

\DeclareAcronym{adas}{
  short = ADAS,
  long = Advanced Driver-Assistance Systems,
}

\DeclareAcronym{it}{
  short = IT,
  long = Information Technology,
}

\DeclareAcronym{etl}{
  short = ETL,
  long = Extract Transform Load,
}

\DeclareAcronym{ram}{
  short = RAM,
  long = Random Access Memory,
}

\DeclareAcronym{hdd}{
  short = HDD,
  long = Hard Disk Drive,
}

\DeclareAcronym{ssd}{
  short = SSD,
  long = Solid State Drive,
}

\DeclareAcronym{cpu}{
  short = CPU,
  long = Central Processing Unit,
}

\DeclareAcronym{haas}{
  short = HaaS,
  long = Hardware as a Service,
}


% The Roman pages, like the Roman Empire, must come to its inevitable close.
\end{romanpages}

% Setting linkcolor here so that Fig & Chap refs are colored when hidelinks=false
% but toc entries are always black
\hypersetup{
    linkcolor=darkblue,
}

%%%%% CHAPTERS
% Add or remove any chapters you'd like here, by file name (excluding '.tex'):
\flushbottom

% all your chapters and appendices will appear here
\hypertarget{introduction}{%
\chapter*{Introduction}\label{introduction}}
\addcontentsline{toc}{chapter}{Introduction}

\adjustmtc

La mia introduzione \ldots{}

--\textgreater{}

\hypertarget{chap:nlip-ita-market}{%
\chapter{\texorpdfstring{\textbf{Non Life Insurance Pricing in the Italian market}}{Non Life Insurance Pricing in the Italian market}}\label{chap:nlip-ita-market}}

\minitoc  

\chaptermark{Non Life Insurance Pricing in the Italian market}

Lorem ipsum dolor sit amet, consectetur adipiscing elit. Vivamus id mauris interdum, malesuada ante eu, tempus lacus. Aliquam blandit tortor a velit ultricies, eget pharetra nulla egestas. Suspendisse pellentesque finibus est, vitae ullamcorper magna convallis ut. Nulla a lectus in ligula iaculis convallis. Pellentesque tortor mauris, tempor nec dictum et, facilisis sit amet dolor. Mauris nibh quam, molestie non ex quis, hendrerit dignissim nulla. Aliquam sit amet dui at diam vestibulum malesuada a id lacus. Phasellus viverra orci vitae sem pretium, eu consequat libero euismod.

Cras suscipit aliquam consequat. Quisque sodales lacus ac erat malesuada, eu laoreet enim vestibulum. Sed id ante id ligula auctor ullamcorper. Sed luctus rutrum mollis. Vestibulum sed ultrices quam. Duis id orci ut enim elementum maximus id quis justo. Pellentesque rutrum ligula in aliquam rhoncus. Integer suscipit nisl at mi efficitur interdum. Aenean et orci elit.

Nam ultricies est et iaculis tempus. Quisque leo lorem, sagittis et ligula a, blandit mattis velit. Phasellus pretium, orci et semper finibus, dui nulla tempor nisl, vel vehicula magna diam nec sem. Praesent finibus commodo enim non laoreet. Lorem ipsum dolor sit amet, consectetur adipiscing elit. Curabitur ut pellentesque purus. Proin hendrerit, odio vel sodales porta, ex lorem feugiat sem, non fringilla libero ex ac ligula. Quisque facilisis eros at suscipit rhoncus.

--\textgreater{}

\hypertarget{chap:models}{%
\chapter{\texorpdfstring{\textbf{Statistical models used in Non Life Insurance Pricing}}{Statistical models used in Non Life Insurance Pricing}}\label{chap:models}}

\minitoc  

\chaptermark{Statistical models used in Non Life Insurance Pricing}

Lorem ipsum dolor sit amet, consectetur adipiscing elit. Vivamus id mauris interdum, malesuada ante eu, tempus lacus. Aliquam blandit tortor a velit ultricies, eget pharetra nulla egestas. Suspendisse pellentesque finibus est, vitae ullamcorper magna convallis ut. Nulla a lectus in ligula iaculis convallis. Pellentesque tortor mauris, tempor nec dictum et, facilisis sit amet dolor. Mauris nibh quam, molestie non ex quis, hendrerit dignissim nulla. Aliquam sit amet dui at diam vestibulum malesuada a id lacus. Phasellus viverra orci vitae sem pretium, eu consequat libero euismod.

Cras suscipit aliquam consequat. Quisque sodales lacus ac erat malesuada, eu laoreet enim vestibulum. Sed id ante id ligula auctor ullamcorper. Sed luctus rutrum mollis. Vestibulum sed ultrices quam. Duis id orci ut enim elementum maximus id quis justo. Pellentesque rutrum ligula in aliquam rhoncus. Integer suscipit nisl at mi efficitur interdum. Aenean et orci elit.

Nam ultricies est et iaculis tempus. Quisque leo lorem, sagittis et ligula a, blandit mattis velit. Phasellus pretium, orci et semper finibus, dui nulla tempor nisl, vel vehicula magna diam nec sem. Praesent finibus commodo enim non laoreet. Lorem ipsum dolor sit amet, consectetur adipiscing elit. Curabitur ut pellentesque purus. Proin hendrerit, odio vel sodales porta, ex lorem feugiat sem, non fringilla libero ex ac ligula. Quisque facilisis eros at suscipit rhoncus.

--\textgreater{}

\hypertarget{chap:practical-app}{%
\chapter{\texorpdfstring{\textbf{Practical application}}{Practical application}}\label{chap:practical-app}}

\minitoc  

\chaptermark{Practical application}

Lorem ipsum dolor sit amet, consectetur adipiscing elit. Vivamus id mauris interdum, malesuada ante eu, tempus lacus. Aliquam blandit tortor a velit ultricies, eget pharetra nulla egestas. Suspendisse pellentesque finibus est, vitae ullamcorper magna convallis ut. Nulla a lectus in ligula iaculis convallis. Pellentesque tortor mauris, tempor nec dictum et, facilisis sit amet dolor. Mauris nibh quam, molestie non ex quis, hendrerit dignissim nulla. Aliquam sit amet dui at diam vestibulum malesuada a id lacus. Phasellus viverra orci vitae sem pretium, eu consequat libero euismod.

Cras suscipit aliquam consequat. Quisque sodales lacus ac erat malesuada, eu laoreet enim vestibulum. Sed id ante id ligula auctor ullamcorper. Sed luctus rutrum mollis. Vestibulum sed ultrices quam. Duis id orci ut enim elementum maximus id quis justo. Pellentesque rutrum ligula in aliquam rhoncus. Integer suscipit nisl at mi efficitur interdum. Aenean et orci elit.

Nam ultricies est et iaculis tempus. Quisque leo lorem, sagittis et ligula a, blandit mattis velit. Phasellus pretium, orci et semper finibus, dui nulla tempor nisl, vel vehicula magna diam nec sem. Praesent finibus commodo enim non laoreet. Lorem ipsum dolor sit amet, consectetur adipiscing elit. Curabitur ut pellentesque purus. Proin hendrerit, odio vel sodales porta, ex lorem feugiat sem, non fringilla libero ex ac ligula. Quisque facilisis eros at suscipit rhoncus.


%%%%% REFERENCES

% JEM: Quote for the top of references (just like a chapter quote if you're using them).  Comment to skip.
% \begin{savequote}[8cm]
% The first kind of intellectual and artistic personality belongs to the hedgehogs, the second to the foxes \dots
%   \qauthor{--- Sir Isaiah Berlin \cite{berlin_hedgehog_2013}}
% \end{savequote}

\setlength{\baselineskip}{0pt} % JEM: Single-space References

{\renewcommand*\MakeUppercase[1]{#1}%
\printbibliography[heading=bibintoc,title={\bibtitle}]}

\end{document}
